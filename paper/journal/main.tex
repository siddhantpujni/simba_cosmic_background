% This a LaTeX template for a research journal, aimed at 
% being 
% 1. easy to use, so one can simply type the daily entries;
% 2. elegant.
% It was written by Níckolas Alves (alves-nickolas.github.io)

% THIS WAS ORIGINALLY COMPILED WITH LUALATEX, so I suggest going to the Menu (top-left of your screen, if you're on Overleaf) and selecting LuaLaTeX on Settings -> Compiler. I didn't test it on XeLaTeX, but I think it should work fine as well. While the document will still compile on pdfLaTeX, for example, it will not have access to the FiraMath font, and hence math text will look weird (it will be typeset in LaTeX's standard Computer Modern Math). If you don't plan on using mathematics at all, then there is a reasonable chance pdfLaTeX will do just fine

\documentclass[a4paper, 11pt, oneside]{researchjournal} % I wrote the design using a4paper, 11pt, oneside, but feel free to change

\usepackage[margin=0.7in]{geometry}
\usepackage{cleveref}

%\logo{} can be used to add a small decoration to the top of the cover page. My original idea was to put an \insergraphics command in it and load, e.g., the university logo or something

\author{Siddhant Pujni\\ % you can use double bars to add lines to the author decoration on the main page
The University of Edinburgh}

% colors are customizable using xcolor's (https://ctan.org/pkg/xcolor) \definecolor
\definecolor{ChapterBackground}{HTML}{101010} %colors to use on chapters
\definecolor{ChapterForeground}{HTML}{2FAEA3} %colors to use on chapters
\definecolor{DayColor}{HTML}{2FAEA3} %colors to use on newdays and daybibs
\definecolor{CoverBackground}{HTML}{101010} %cover background
\definecolor{CoverForeground}{HTML}{2FAEA3} %cover letters
\definecolor{LinkColor}{HTML}{2FAEA3} %color for links

\setlength\parindent{0pt}

\begin{document} % this will automatically generate a simple cover

\newday*{2026-01-15} 
Had the preliminary meeting with Romeel yesterday to get going on the project. Had already contacted and set-up required Python packages and Linux. Was given an article and a paper that is close to the project. They are more observational, and we will be doing something similar, but for the SIMBA simulations. CAESAR takes snapshots of the simulation of all galaxies at a specific redshift and outputs their properties, including absolute magnitude, apparent magnitude, etc. We will meet every Wednesday at 11 am, and I will prepare a small set of slides for these meetings to cover what I have achieved and done over the last week; mostly figures and images, just to refer to for report writing. 

\vspace{0.5em}

The paper discusses \cite{hillSpectrumUniverse2018}.

\vspace{0.5em}

The article discusses \cite{bakerWhatColorUniverse2021}.

Need to do more reading regarding and learn about:
\begin{itemize}
\itemsep -6pt {}
    \item CAESAR documentation (specifically Photometry section)
    \item FSPS
    \item H5py for file reading
    \item also set up the GitHub repo for organised work
\end{itemize}

Also assigned the first task, which is to replicate Figure 4 from \cite{hillSpectrumUniverse2018}, but just for one snapshot of the SIMBA simulation. The goal is to essentially look at all galaxies at some redshift and sum up their fluxes for each band, then plot them by extracting wavelength and magnitudes. Adding up all intensities for each filter and then plotting each filter at its respective frequency.

\newday{2026-01-16}
Goals for the day:
\begin{itemize}
\itemsep -6pt {}
    \item Set up the GitHub repo 
    \item Read necessary documentations
    \item Figure out what exactly the graph is plotting and what $\nu I_{\nu}$ is. 
\end{itemize}

Using the CAESAR Catalogue with 25 Mpc/h volume and $2\times256^3$ particles, at z=0.

Spent some time setting up the GitHub repo with working references from zotero and Overleaf linking for report writing and referencing. Everything now works as one big thing together.

\newday{2026-01-17} Worked on building the SED of all galaxies in all bands from the Caesar catalogue. The catalogue was loaded via caesar.load, and a basic histogram in \cref{fig:mass_diss} of log10(total mass) by extracting each galaxy’s total mass (i.masses['total']) was used to visualise the overall galaxy population. Some time was spent to learn and familiarising with the overall file and data structure of the hdf5 files in general.
 
\begin{figure}[htbp]
    \centering
    \includegraphics[width=0.7\textwidth]{figures/week_1/mass_diss.png}
    \caption{Histogram of total galaxy mass from the SIMBA z=0 snapshot.}
    \label{fig:mass_diss}
\end{figure}

\newpage 

Essentially constructed a catalogue-wide SED from stored apparent magnitudes accounting for dust (galaxy\_data/dicts/appmag.*): for each filter it converts magnitudes to flux density using the AB relation $F_\nu = 3631,\mathrm{Jy},10^{-m/2.5}$, obtains an effective frequency ($\nu$) from the FSPS filter’s effective wavelength $(\nu=c/\lambda_\mathrm{eff})$, sums flux over all galaxies to get a total per-filter $F_\nu$, sorts by frequency, and plots $\nu F_\nu$ versus $\nu$ on log–log axes to show the overall SED shape of the sample across the available bands in \cref{fig:sed_z0}.

\begin{figure}[htbp]
    \centering
    \includegraphics[width=0.7\textwidth]{figures/week_1/galaxy_sed.png}
    \caption{Spectral energy distribution from the SIMBA z=0 snapshot, plotted as $\nu F_{\nu}$ versus frequency $\nu$ (assuming AB magnitudes).}
    \label{fig:sed_z0}
\end{figure}

\newpage

\newday{2026-01-20} Setting up Cuillin login and accessing Simba data through it, needed to learn some linux commands to navigate through directories and files. Managed to access the Simba data and set up a symlink in the home directory to access it for analysis. The goal for the rest of the week is to create the same plot at different redshifts to examine how things evolve over time. Also aiming to break up the SED into different components -- e.g., star-forming galaxies vs quenched galaxies, a range of stellar mass bins to gain a sense of what objects are dominating the overall cosmic light at given wavelengths and redshifts. I need to add more references and papers about the tools that I'm using.

\newday{2026-01-22} Working on looking at the time evolution of full galaxy SED across different redshifts. No galaxies have formed above a redshift of z=? (below file number 019) and its only halos so there is no SED to be made. Also will be ignoring snap 151 from now on which is effectively z=0 because apparent = abs magnitude because its at 0 distance from us.

Created a plot in \cref{fig:time_evo_abs_mag_sed} and \cref{fig:time_evo_app_mag_sed} showing the SED at different redshifts from z~9 to z~0 for both apparent and absolute magnitudes just taking occasional redshift jumps. my general understanding for the absolute magnitude SEDs was that as time goes on (lower redshifts) it increases in intensity because more stars and galaxies form over time contributing to overall light. for the apparent magnitude it was more so that we're observing a lot more flux from closer galaxies so higher intensity and is mainly showcasing the distance effects. As for the overall shape my understanding was its determined by hotter bluer stars which contributes to fluxes at shorter wavelengths and the longer redder wavelengths are dominated by older cooler stars the combination of which will make up the overall SED. Interestingly though in the absolute mags the lowest z are not the very highest curves.  but of course in apparent mags they are farther away so they appear fainter. The general shape also changes with redshift, which has to do with how many older vs younger stars there are.

\begin{figure}[htbp]
    \centering
    \includegraphics[width=0.8\textwidth]{figures/week_2/time_evo_abs_mag_SED.png}
    \caption{Time evolution of the spectral energy distribution of absolute magnitudes from the SIMBA snapshots, plotted as $\nu F_{\nu}$ versus wavelength $\lambda$ (assuming AB magnitudes).}
    \label{fig:time_evo_abs_mag_sed}
\end{figure}

\begin{figure}[hbt!]
    \centering
    \includegraphics[width=0.8\textwidth]{figures/week_2/time_evo_app_mag_SED.png}
    \caption{Time evolution of the spectral energy distribution of apparent magnitudes from the SIMBA snapshots, plotted as $\nu F_{\nu}$ versus wavelength $\lambda$ (assuming AB magnitudes).}
    \label{fig:time_evo_app_mag_sed}
\end{figure}

\newpage

\newday{2026-01-23} Aiming to dive deeper and explore the above graphs by taking one of the redshifts and splitting it into bins of various galaxy types, e.g. star forming galaxies vs quenched galaxies, different stellar mass bins, etc.
Going to be defining quenched galaxies using sfr (instantaneous), sfr\_100 (avg sfr over last 100My), and ssfr ($\frac{\text{sfr}}{M_*}$, $M_*$ is total galaxy stellar mass).





\end{document}



\daybib\cite{weinberg1995Foundations,weinberg1996ModernApplications}. %daybib adds the text "References: " underneath the entry. It just prints text without doing anything fancy. I use it to list references that I used on some given day, but didn't make it to the main paragraph. Notice I manually added a period at the end of the line.

\newday{2023-03-03} References are dealt with using \verb|biblatex|. You can add your own my modifying the file \verb|bib.bib|.
